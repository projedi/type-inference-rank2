\frame {
   \frametitle{System F}
   \begin{align*}
      \mathbb{T} ::=&\ \mathbb{V}\\
                   |&\ \mathbb{T} \rightarrow \mathbb{T}\\
                   |&\ \forall \mathbb{V}.\ \mathbb{T}
   \end{align*}
   Сильная нормализуемость $\Rightarrow$ типизируемость неразрешима.
   \pause

   Ранг типа - насколько глубоко может находиться $\forall$ по левую сторону $\rightarrow$.
   \cite{KW93} доказали разрешимость для рангов $\leq 2$.
}

\frame {
   \frametitle{Но зачем?}
   \begin{itemize}
      \item<1-> Классический пример:
         \(\lambda f.\ f\ f\uncover<3->{:: \color<3->{red}{\forall b.\ (\forall a.\ a) \rightarrow b}}\)
      \item<2-> Church numerals:
         \begin{align*}
            \mathtt{FALSE} &:= \lambda x y.\ y \uncover<3->{:: \color<3>{blue}{
               \forall a.\ a \rightarrow a \rightarrow a}}\\
            \mathtt{TRUE}  &:= \lambda x y.\ x \uncover<3->{:: \color<3>{blue}{
               \forall a.\ a \rightarrow a \rightarrow a}}\\
            n &:= \lambda f x.\ f^n\ x \uncover<3->{::\color<3>{blue}{
               \forall a.\ (a \rightarrow a) \rightarrow a \rightarrow a}}\\
            iszero &:= \lambda n.\ n\ (\lambda x.\ \mathtt{FALSE})\ \mathtt{TRUE} \\
                   &\uncover<3->{:: \color<3->{red}{
                      \forall b.\ (\forall a.\ (a \rightarrow a) \rightarrow a \rightarrow a)
                      \rightarrow b \rightarrow b \rightarrow b}}
         \end{align*}
   \end{itemize}
   \uncover<4->{ Практическое применение \cite{SPJ07}: }
   \begin{itemize}
      \item<4-> Инкапсуляция состояния:
         \( runST :: \color<4->{red}{\forall a.\ (\forall s.\ ST\ s\ a) \rightarrow a}\)
      \item<5-> Динамическая типизация(Закон Лейбница): %TODO: Elaborate
         \begin{align*}
            &\mathbf{data}\ Equal\ a\ b\ \mathbf{where}\\
            &\ Equal :: \color<5->{red}{(\forall f.\ f\ a \rightarrow f\ b) \rightarrow Equal\ a\ b}
         \end{align*}
   \end{itemize}
   \uncover<6-> {\em Все представленные типы ранга 2.}
}

\frame {
   \frametitle{Алгоритмы вывода типов ранга 2}
   \begin{enumerate}[<+->]
      \item \alt<-1>{Nancy McCracken, 1984\\
         {\em The typechecking of programs with implicit type structure}.
         }{\sout{Nancy McCracken, 1984\\
         {\em The typechecking of programs with implicit type structure}.
         }}
         \uncover<2->{\textbf{ОШИБОЧЕН}\cite{Jim95}.}
      \item A. J. Kfoury, J. B. Wells, 1993\\
         {\em A Direct Algorithm for Type Inference in the Rank-2 Fragment
          of the Second-Order $\lambda$-Calculus}. \\
         Основной алгоритм в данной работе.
      \item Brad Lushman, Gordon V. Cormack, 2007\\
         {\em A More Direct Algorithm for Type Inference in the Rank-2 Fragment
          of the Second-Order $\lambda$-Calculus}. \\
         Основан на \cite{KW93}, вычислительно эффективнее.
   \end{enumerate}
}

\frame {
   \frametitle{Kfoury, Wells алгоритм}
   \begin{enumerate}[<+->]
      \item \textbf{Входные данные}.
         \begin{align*}
            &\text{Терм } N,\quad
            \mathtt{FTV}(N) = \overrightarrow{\omega},\quad
            \mathtt{ACT}(N) = \overrightarrow{x}\\
            \only<1>{
               %&\mathtt{ACT}(\lambda x_1 x_2.\ (\lambda x_3 x_4.\ x_4) (\lambda x_5.\ x_5)) = \{x_1, x_2, x_4\}
               &\mathtt{ACT}(\lambda x_1 x_2.\ \mathbf{let}\ x_3 = \lambda x_5.\ x_5\ \mathbf{in}\ \lambda x_4.\ x_4) = \{x_1, x_2, x_4\}
               \\
            }
            &\overrightarrow{x} :: \overrightarrow{\tau_x},\quad
            \overrightarrow{\omega} :: \overrightarrow{\tau_{\omega}},\quad
            \text{все }\tau\text{ ранга }1
         \end{align*}
         \only<1>{
         Если для $x$ или $\omega$ не предоставить тип, то подразумевается $\forall a.\ a$.
         }
      \item \textbf{$\theta$-редукция}. \only<2>{Преобразование $N$ к виду:
         \begin{align*}
            \lambda \overrightarrow{x}.\ 
            \mathbf{let}\ &y_1 = T_1(\overrightarrow{x},\overrightarrow{\omega}) \\
                          &y_2 = T_2(\overrightarrow{x},\overrightarrow{\omega},y_1) \\
                          &\ \vdots \\
                          &y_k = T_k(\overrightarrow{x},\overrightarrow{\omega},y_1,\dots,y_{k-1}) \\
            \mathbf{in}\ &T_{k+1}(\overrightarrow{x},\overrightarrow{\omega},y_1,\dots,y_k)
         \end{align*} }
      \item \textbf{Acyclic Semi-Unification Problem}. \only<3>{
         %TODO: Simplify
         \[
            \tau \leq \sigma \Leftrightarrow S(\tau) = \sigma,\quad
            S(\mathbb{T} \rightarrow \mathbb{T}) = S(\mathbb{T}) \rightarrow S(\mathbb{T}),\quad
            S(\mathbb{V}) \in \mathbb{T}
         \]
         Semi-Unification problem:
         \[
            \{(\tau_i,\sigma_i)\} \mapsto S: \forall i.\ \text{ выполняется } S(\tau_i) \leq S(\sigma_i) 
         \]
         %TODO: Define tranformation from theta-normal form to ASUP
         Неразрешима, но Acyclic Semi-Unification problem разрешима.  }
      \item \textbf{Выводимый тип}. \only<4>{
         \[
            \forall (\mathtt{FTV}(\sigma) \setminus
               (\mathtt{FTV}(\overrightarrow{\tau_x}) \cup \mathtt{FTV}(\overrightarrow{\tau_\omega} )).\ 
               \tau_{x_1} \rightarrow \tau_{x_2} \rightarrow
               \ldots \rightarrow \tau_{x_n} \rightarrow \sigma,
         \]
         где $\sigma$ ранга 0.  }
   \end{enumerate}
}

   %\[
   %TODO: Do I need this ASUP definition?
   %\begin{array}{r@{\ \leq\ } l | r@{\ \leq\ } l | c | r@{\ \leq\ } l}
      %\tau_{1,1} & \sigma_{1,1} & \tau_{1,2} & \sigma_{1,2} &
         %\cdots & \tau_{1,n} & \sigma_{1,n} \\

      %\tau_{k,1} & \sigma_{k,1} & \tau_{k,2} & \sigma_{k,2} &
         %\cdots & \tau_{k,n} & \sigma_{k,n} \\[\dimexpr-\normalbaselineskip+1ex]
      %\multicolumn{1}{l}{\underbrace{}_{V_0}} &
      %\multicolumn{2}{@{}l@{}}{\underbrace{\hspace*{\dimexpr2\tabcolsep}\hphantom{\sigma{k,1} \tau_{k,2}}}_{V_1}} &
      %\multicolumn{3}{@{}l@{}}{\underbrace{\hspace*{\dimexpr2\tabcolsep}\hphantom{\sigma{k,2} \dots \tau{k,n}}}_{\dots}} &
      %\multicolumn{1}{@{}l}{\underbrace{}_{V_{n+1}}}
   %\end{array}\quad 
      %V_i \cap V_j = \varnothing,\quad i \neq j
   %\]

\frame {
   \frametitle{Проблемы алгоритма}
   \begin{itemize}[<+->]
      \item $\theta$-редукция изменяет терм $\Rightarrow$ не установить, в какой
            части оригинального терма произошла ошибка типизации.
      \item Свободные переменные обязаны быть ранга 1 $\Rightarrow$ тип терма выводим,
            но использовать в других термах его не можем.
      \item Типы аргументов и свободных переменных не выводимы $\Rightarrow$
            типы аргументов необходимо указывать и вывод типов при взаимной рекурсии невозможен.
   \end{itemize}
}

\frame {
   \frametitle{Фундаментальная проблема System F}
   %TODO: Simplify
   Понятие наиболее общего типа не определено: нету ни главного
   типа, ни главной пары.

   Главный тип($\sigma$):
   \[
      \exists A, \tau.\ A \vdash M: \tau \Rightarrow \exists \sigma.\ A \vdash M: \sigma,
   \]
   где $\sigma$ представляет все возможные типизации $M$ в $A$.

   Главная пара($(B, \sigma)$):
   \[
      \exists A, \tau.\ A \vdash M: \tau \Rightarrow \exists B, \sigma.\ B \vdash M: \sigma,
   \]
   где $(B, \sigma)$ представляет все возможные типизации $M$.
   \pause
   \begin{align*}
      \lambda f.\ f\ f &:: (\forall a.\ a) \rightarrow (\forall b.\ b) \\
      \lambda f.\ f\ f &:: (\forall a.\ a \rightarrow a) \rightarrow (\forall b.\ b \rightarrow b)
   \end{align*}
}

\frame {
   \frametitle{Преимущества главной пары}
   \begin{description}
      \item[Модульная компиляция] терм нужно перекомпилировать только, когда терм
         изменяется, а не когда изменяется терм одной из свободной переменной.
      \item[Добавление рекурсии]
      \[
         \begin{array}{l}
            \AxiomC{$A \cup \{ x :: \tau \} \vdash M :: \tau$}
            \RightLabel{($\tau$ ранга 0)}
            \UnaryInfC{$A \vdash (\mu x.\ M) :: \tau$}
            \DisplayProof \\
            \\

            \AxiomC{$A \cup \{ x :: \tau \} \vdash M :: \sigma$}
            \RightLabel{($\sigma \leq \tau$)}
            \UnaryInfC{$A \vdash (\mu x.\ M) :: \sigma$}
            \DisplayProof
         \end{array}
      \]
      \item[Сообщения об ошибках] $(\lambda x.\ M) N$ \\
         Типизируем отдельно $M$ и $N$ $\Rightarrow$ можно отличить ошибки в $M$,
         вызванные не $N$ и сузить область поиска ошибки.
         %Главная пара для $M$ -> тип для $x$ -> проверяем подходит ли главный тип для $N$.
         %Главный тип для $N$ -> подставляем в $M$ -> ошибки ....
      %TODO: Tell about incremental type inference(for REPL)
   \end{description}
}

\frame {
   \frametitle{Что же делать?}
   \begin{itemize}
      \item \cite{SPJ07} предлагает алгоритмы для работы с произвольным рангом, но
         которым требуются подсказки для некоторых термов.
      \item \cite{Jim95} предлагает использовать типы-пересечения, поскольку они
         типизируют ровно те же термы, но обладают главной парой
      %TODO: What are intersection types?
      %\item Попробовать поискать альтернативные представления общего типа.
         %\cite{Jim95} предлагает использовать пересечения в качестве такого представления
   \end{itemize}
}

\frame {
   \frametitle{What have I done?}
   Реализация алгоритма \cite{KW93} \url{https://github.com/projedi/type-inference-rank2}
}

\frame {
   \frametitle{The bibliography}
   \begin{thebibliography}{Kfoury, Wells, 1993}
      \bibitem[Kfoury, Wells, 1993]{KW93}
         A. J. Kfoury, J. B. Wells
         \newblock{\em A Direct Algorithm for Type Inference in the Rank-2 fragment of the Second Order $\lambda$-calculus}
         \newblock 1994
      \bibitem[Jim, 1995]{Jim95}
         T. Jim
         \newblock{\em What are principal typings and what are they good for?}
         \newblock 1995
      \bibitem[Peyton Jones, 2007]{SPJ07}
         S. P. Jones, D. Vytiniotis, S. Weirich, M. Shields
         \newblock {\em Practical Type Inference for arbitrary-rank types}.
         \newblock 2007
      \bibitem[The Church Project]{ChurchProject}
         \url{http://types.bu.edu/}
         \newblock {\em The Church Project}
         \newblock последнее обновление в 2010
   \end{thebibliography}
}

\frame {
   \frametitle{That's all, folks!}
   Questions?
}
