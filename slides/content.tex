\frame {
   \frametitle{Зачем нужны полиморфные типы?}
   \begin{align*}
      \mathbb{T} ::=&\ \mathbb{V}\\
                   |&\ \mathbb{T} \rightarrow \mathbb{T}
   \end{align*}
   \pause
   \begin{itemize}[<+->]
      \item Классический пример:
         \(\lambda f.\ f\ f\)
   \end{itemize}
}

\frame {
   \frametitle{System F}
   \[\begin{array}{rlr}
      \mathbb{T} ::=&\ \mathbb{V} \quad & \multirow{3}{*}{
         \AxiomC{$A \vdash M : \forall \alpha. \tau $}
         \LeftLabel{INST}
         \UnaryInfC{$A \vdash M : \tau[\alpha := \sigma]$}
         \DisplayProof
      }\\
       |&\ \mathbb{T} \rightarrow \mathbb{T}\\
       |&\ \color<1>{red}{\forall \mathbb{V}.\ \mathbb{T}}
   \end{array}\]
   \pause
   Типизируемость неразрешима.
   \pause

   Ранг типа - насколько глубоко может находиться $\forall$ по левую сторону $\rightarrow$.

   \cite{KW93} доказали разрешимость для рангов $\leq 2$.
}

\frame {
   \frametitle{System F}
   \begin{itemize}
      \item Классический пример:
         \(\lambda f.\ f\ f : \color{red}{\forall b.\ (\forall a.\ a) \rightarrow b}\)
   \end{itemize}
   \pause
   Практическое применение \cite{SPJ07}:
   \begin{itemize}[<+->]
      \item Инкапсуляция состояния:
         \( runST : \color{red}{\forall a.\ (\forall s.\ ST\ s\ a) \rightarrow a}\)
      \item Динамическая типизация:
         \begin{align*}
            &\mathbf{data}\ Equal\ a\ b\ \mathbf{where}\\
            &\ \ \ Equal : \color{red}{(\forall f.\ f\ a \rightarrow f\ b) \rightarrow Equal\ a\ b}
         \end{align*}
         \only<3-4>{
         Закон Лейбница: \(a \equiv b = \forall f.\ f\ a \alt<3>{\textcolor{blue}{\ \Leftrightarrow\ }}
                                                           {\textcolor<4>{red}{\rightarrow\ }} f\ b\).
         }
   \end{itemize}
   \uncover<5>{\em Все представленные типы ранга 2.}
}

\frame {
   \frametitle{Алгоритмы вывода типов ранга 2}
   \begin{enumerate}
      \item<1-> \alt<-1>{Nancy McCracken, 1984\\
         {\em The Typechecking of Programs With Implicit Type Structure}.
         }{\sout{Nancy McCracken, 1984\\
         {\em The Typechecking of Programs With Implicit Type Structure}.
         }}
         \uncover<2->{\textbf{ОШИБОЧЕН}\cite{Jim95}.}
      \item<3-> A. J. Kfoury, J. B. Wells, 1993\\
         {\em A Direct Algorithm for Type Inference in the Rank-2 Fragment
          of the Second-Order $\lambda$-Calculus}. \\
         Основной алгоритм в данной работе.
      \item<4-> Brad Lushman, Gordon V. Cormack, 2007\\
         {\em A More Direct Algorithm for Type Inference in the Rank-2 Fragment
          of the Second-Order $\lambda$-Calculus}. \\
         Основан на \cite{KW93}, вычислительно эффективнее.
   \end{enumerate}
}

\frame {
   \frametitle{Kfoury, Wells алгоритм}
   \begin{enumerate}
      \item[0.]<1-> \textbf{Входные данные}.
         \begin{align*}
            &\text{Терм } N,\quad
            \mathtt{FTV}(N) = \overrightarrow{\omega},\quad
            \mathtt{ACT}(N) = \overrightarrow{x} %TODO: ACT means active abstractions
            \only<3->{\\
            &\overrightarrow{x} : \overrightarrow{\tau_x},\quad
            \overrightarrow{\omega} : \overrightarrow{\tau_{\omega}},\quad
            \text{все }\tau\text{ ранга }1
            }
         \end{align*}
         \only<1-2>{
            %TODO: A bit more challenging term?
            \[
            \alt<1>{
               \mathtt{ACT}(\lambda \textcolor{red}{x_1 x_2}.\ 
               \mathbf{let}\ x_3 = (\lambda x_5.\ x_5)\ 
               \mathbf{in}\ (\lambda \textcolor{red}{x_4}.\ x_4)) = \{x_1, x_2, x_4\}
            }{
               \mathtt{ACT}(\lambda x_1 x_2.\ \lambda x_4.\ x_4) = \{x_1, x_2, x_4\}
            }
            \]
         }
         \only<3>{
            Если для $x$ или $\omega$ не предоставить тип, то подразумевается $\bot = \forall a.\ a$.
         }
      \item<4-> \textbf{$\theta$-редукция}. \alt<4>{Преобразование $N$ к виду:
         \begin{align*}
            \lambda \overrightarrow{x}.\ 
            \mathbf{let}\ &y_1 = T_1(\overrightarrow{x},\overrightarrow{\omega}) \\
                          &y_2 = T_2(\overrightarrow{x},\overrightarrow{\omega},y_1) \\
                          &\ \vdots \\
                          &y_k = T_k(\overrightarrow{x},\overrightarrow{\omega},y_1,\dots,y_{k-1}) \\
            \mathbf{in}\ &T_{k+1}(\overrightarrow{x},\overrightarrow{\omega},y_1,\dots,y_k)
         \end{align*}}{
            \(\lambda \overrightarrow{x}.\ (\lambda y_1.\ (\dots (\lambda y_k.\ T_{k+1}) T_k \dots)) T_1 \)
         }
      \item<5-> \textbf{Acyclic Semi-Unification Problem}. \only<5>{
            \[\begin{array}{l}
               \tau \leq \sigma \Leftrightarrow S(\tau) = \sigma\\
               S(\sigma \rightarrow \tau) = S(\sigma) \rightarrow S(\tau),\quad
               S(\mathbb{V}) \in \mathbb{T}\\
               \tau, \sigma \text{ ранга 0}
            \end{array}\]
         }\only<6-7>{
            \begin{align*}
               \sigma \doteq \tau &\Leftrightarrow \alpha \rightarrow \alpha \leq' \sigma \rightarrow \tau\\
               \tau_x &= \forall \overrightarrow{\alpha}. \tau_x',\ \tau_x' \text{ ранга 0},\ \dots\ \omega\\
               \tau'_x &\leq' \beta_{x,0},\quad \tau'_{\omega} \leq' \beta_{\omega,0},
            \end{align*}
         }\only<7>{
            \\Неравенства для $T_i$:
            \begin{align*}
               N\ M &\mapsto \delta_N \doteq \delta_M \rightarrow \delta_{N M}\\
               \lambda z.\ N &\mapsto \delta_{\lambda z.\ N} \doteq \gamma_z \rightarrow \delta_N\\
               z &\mapsto \gamma_z \doteq \delta_z\\
               x &\mapsto \beta_{x,i-1} \leq' \delta_x,\ \dots\ \omega,\ \dots\ y\\
               \beta_{x,i-1} &\leq' \beta_{x,i},\ \dots\ \omega,\ \dots\ y
            \end{align*}
         }\only<8-9>{\\
            Semi-Unification problem:
            \[
               \{\tau_i \leq' \sigma_i\} \mapsto S: \forall i.\ S(\tau_i) \leq S(\sigma_i) 
            \]
         }\only<9>{
            Неразрешима, но ограниченный вариант - Acyclic Semi-Unification problem - разрешима.
         }\only<10->{ \(S,\ \delta_{T_{k+1}}\) }
      \item<10-> \textbf{Выводимый тип}.
         \[
            \tau_{x_1} \rightarrow \ldots \rightarrow \tau_{x_n} \rightarrow S(\delta_{T_{k+1}})
         \]
   \end{enumerate}
}

   %\[
   %TODO: Do I need this ASUP definition?
   %\begin{array}{r@{\ \leq\ } l | r@{\ \leq\ } l | c | r@{\ \leq\ } l}
      %\tau_{1,1} & \sigma_{1,1} & \tau_{1,2} & \sigma_{1,2} &
         %\cdots & \tau_{1,n} & \sigma_{1,n} \\

      %\tau_{k,1} & \sigma_{k,1} & \tau_{k,2} & \sigma_{k,2} &
         %\cdots & \tau_{k,n} & \sigma_{k,n} \\[\dimexpr-\normalbaselineskip+1ex]
      %\multicolumn{1}{l}{\underbrace{}_{V_0}} &
      %\multicolumn{2}{@{}l@{}}{\underbrace{\hspace*{\dimexpr2\tabcolsep}\hphantom{\sigma{k,1} \tau_{k,2}}}_{V_1}} &
      %\multicolumn{3}{@{}l@{}}{\underbrace{\hspace*{\dimexpr2\tabcolsep}\hphantom{\sigma{k,2} \dots \tau{k,n}}}_{\dots}} &
      %\multicolumn{1}{@{}l}{\underbrace{}_{V_{n+1}}}
   %\end{array}\quad 
      %V_i \cap V_j = \varnothing,\quad i \neq j
   %\]

\frame {
   \frametitle{Проблемы алгоритма}
   \begin{itemize}[<+->]
      \item $\theta$-редукция изменяет терм $\Rightarrow$ не установить, в какой
            части оригинального терма произошла ошибка типизации.
            \only<1>{
            \begin{align*}
               &a\ (\lambda x.\ \llet y = x\ \iin y\ y)\\
               \\
               &\llet v = \lambda x.\ x \\
               &\iin a\ (\lambda x.\ (v\ x) (v\ x))
            \end{align*}}
      \item Свободные переменные обязаны быть ранга 1 $\Rightarrow$ тип терма выводим,
            но использовать в других термах его не можем.
      \item Типы аргументов и свободных переменных не выводимы $\Rightarrow$
            типы аргументов необходимо указывать.
      %TODO: невыводимость аргументов -> их тип _|_ -> честных термов таких нет.
   \end{itemize}
}

\frame {
   \frametitle{Фундаментальная проблема System F}
   Понятие наиболее общего типа не известно. В частности, нет ни главного
   типа, ни главной пары.
   \pause

   Главный тип($\sigma$):
   \begin{align*}
      &\exists A, \tau.\ A \vdash M: \tau \Rightarrow \exists \sigma.\ A \vdash M: \sigma,\\
      &\forall \rho.\ A \vdash M: \rho \Rightarrow \exists S.\ \rho = S(\sigma)
   \end{align*}
   \pause

   Главная пара($(B, \sigma)$):
   \begin{align*}
      &\exists A, \tau.\ A \vdash M: \tau \Rightarrow \exists B, \sigma.\ B \vdash M: \sigma,\\
      &\forall C, \rho.\ C \vdash M: \rho \Rightarrow \exists S.\ \rho = S(\sigma), C = S(B)
   \end{align*}
   \pause

   Пример:
   \begin{align*}
      \vdash \lambda f.\ f\ f &: (\forall a.\ a) \rightarrow (\forall b.\ b) \\
      \vdash \lambda f.\ f\ f &: (\forall a.\ a \rightarrow a) \rightarrow (\forall b.\ b \rightarrow b)
   \end{align*}
}

\frame {
   \frametitle{Преимущества главной пары}
   \begin{itemize}
      \item<1-> \textbf{Модульная компиляция}\only<1>{\\
         Терм нужно перекомпилировать только, когда он изменяется, а не когда
         изменяется терм одной из его свободных переменных.}
      \item<2-> \textbf{Добавление рекурсии}\only<2-7>{
            \[\mu x.\ M = \mathtt{FIX}\ (\lambda x.\ M)\]
         }\only<3-7>{
            \begin{center}
               \AxiomC{$A \cup \{ x : \alt<6->{\textcolor<6>{red}{\sigma}}{\textcolor<5->{blue}{\tau}} \} \vdash M : \tau$}
               \only<3-4>{
                  \LeftLabel{REC-POLY}
                  \RightLabel{($\tau$ ранга \textcolor<4>{blue}{1})}
               }\only<5>{
                  \LeftLabel{REC-SIMPLE}
                  \RightLabel{($\tau$ ранга \textcolor{red}{0})}
               }\only<6->{
                  \LeftLabel{REC}
                  \RightLabel{(\textcolor<6>{red}{$\tau \leq \sigma$})}
               }
               \UnaryInfC{$A \vdash (\mu x.\ M) : \tau$}
               \DisplayProof
            \end{center}
            \only<4>{\textbf{Неразрешимо.}\cite{Jim95}}
         } \only<7>{
            Вместе с модульной компиляцией дают REPL с поддержкой взаимной рекурсии.
         }
      \item<8-> \textbf{Сообщения об ошибках} $(\lambda x.\ M)\ N$ \\
         Типизируем $M$, в котором все $x$ занумерованы, и отдельно $N$.
         Проверяем \(typeof(N) \leq typeof(x_i)\).
         \only<9>{\\
         Без главной пары ошибка может быть обнаружена внутри $N$.
         Пример на Haskell: \[ f = \llet x = (\lambda t \rightarrow t \cdot 3)\ \iin (x\ 2, x\ True) \]
         }
   \end{itemize}
}

\frame {
   \frametitle{Альтернативы}
   \begin{itemize}[<+->]
      \item \cite{SPJ07} предлагает алгоритмы для работы с произвольным конечным
         рангом, но которым требуются подсказки для некоторых термов.
      \item \cite{Jim95} предлагает использовать типы-пересечения, поскольку
         есть системы, типизирующие ровно те же термы, но обладающие главной парой
   \end{itemize}
}

%TODO: If I have time - what are intersection types

\frame {
   \frametitle{Литература}
   \begin{thebibliography}{Kfoury, Wells, 1993}
      \bibitem[Kfoury, Wells, 1993]{KW93}
         A. J. Kfoury, J. B. Wells
         \newblock{\em A Direct Algorithm for Type Inference in the Rank-2 Fragment of the Second Order $\lambda$-Calculus}
         \newblock 1994
      \bibitem[Jim, 1995]{Jim95}
         T. Jim
         \newblock{\em What Are Principal Typings And What Are They Good For?}
         \newblock 1995
      \bibitem[Peyton Jones, 2007]{SPJ07}
         S. P. Jones, D. Vytiniotis, S. Weirich, M. Shields
         \newblock {\em Practical Type Inference for Arbitrary-Rank Types}.
         \newblock 2007
      \bibitem[The Church Project]{ChurchProject}
         \setbeamercolor{bibliography entry author}{fg=blue}\curl{http://types.bu.edu/}
         \newblock {\em The Church Project}
         \newblock последнее обновление в 2010
   \end{thebibliography}
}

\frame {
   \frametitle{Практика}
   Реализация алгоритма \cite{KW93}
   \curl{https://github.com/projedi/type-inference-rank2}
}

\frame {
   \center \Huge Q\&A
}
